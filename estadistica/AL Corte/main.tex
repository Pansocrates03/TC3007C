% Options for packages loaded elsewhere
\PassOptionsToPackage{unicode}{hyperref}
\PassOptionsToPackage{hyphens}{url}
%
\documentclass[
]{article}
\usepackage{amsmath,amssymb}
\usepackage{iftex}
\ifPDFTeX
  \usepackage[T1]{fontenc}
  \usepackage[utf8]{inputenc}
  \usepackage{textcomp} % provide euro and other symbols
\else % if luatex or xetex
  \usepackage{unicode-math} % this also loads fontspec
  \defaultfontfeatures{Scale=MatchLowercase}
  \defaultfontfeatures[\rmfamily]{Ligatures=TeX,Scale=1}
\fi
\usepackage{lmodern}
\ifPDFTeX\else
  % xetex/luatex font selection
\fi
% Use upquote if available, for straight quotes in verbatim environments
\IfFileExists{upquote.sty}{\usepackage{upquote}}{}
\IfFileExists{microtype.sty}{% use microtype if available
  \usepackage[]{microtype}
  \UseMicrotypeSet[protrusion]{basicmath} % disable protrusion for tt fonts
}{}
\makeatletter
\@ifundefined{KOMAClassName}{% if non-KOMA class
  \IfFileExists{parskip.sty}{%
    \usepackage{parskip}
  }{% else
    \setlength{\parindent}{0pt}
    \setlength{\parskip}{6pt plus 2pt minus 1pt}}
}{% if KOMA class
  \KOMAoptions{parskip=half}}
\makeatother
\usepackage{xcolor}
\usepackage[margin=1in]{geometry}
\usepackage{color}
\usepackage{fancyvrb}
\newcommand{\VerbBar}{|}
\newcommand{\VERB}{\Verb[commandchars=\\\{\}]}
\DefineVerbatimEnvironment{Highlighting}{Verbatim}{commandchars=\\\{\}}
% Add ',fontsize=\small' for more characters per line
\usepackage{framed}
\definecolor{shadecolor}{RGB}{248,248,248}
\newenvironment{Shaded}{\begin{snugshade}}{\end{snugshade}}
\newcommand{\AlertTok}[1]{\textcolor[rgb]{0.94,0.16,0.16}{#1}}
\newcommand{\AnnotationTok}[1]{\textcolor[rgb]{0.56,0.35,0.01}{\textbf{\textit{#1}}}}
\newcommand{\AttributeTok}[1]{\textcolor[rgb]{0.13,0.29,0.53}{#1}}
\newcommand{\BaseNTok}[1]{\textcolor[rgb]{0.00,0.00,0.81}{#1}}
\newcommand{\BuiltInTok}[1]{#1}
\newcommand{\CharTok}[1]{\textcolor[rgb]{0.31,0.60,0.02}{#1}}
\newcommand{\CommentTok}[1]{\textcolor[rgb]{0.56,0.35,0.01}{\textit{#1}}}
\newcommand{\CommentVarTok}[1]{\textcolor[rgb]{0.56,0.35,0.01}{\textbf{\textit{#1}}}}
\newcommand{\ConstantTok}[1]{\textcolor[rgb]{0.56,0.35,0.01}{#1}}
\newcommand{\ControlFlowTok}[1]{\textcolor[rgb]{0.13,0.29,0.53}{\textbf{#1}}}
\newcommand{\DataTypeTok}[1]{\textcolor[rgb]{0.13,0.29,0.53}{#1}}
\newcommand{\DecValTok}[1]{\textcolor[rgb]{0.00,0.00,0.81}{#1}}
\newcommand{\DocumentationTok}[1]{\textcolor[rgb]{0.56,0.35,0.01}{\textbf{\textit{#1}}}}
\newcommand{\ErrorTok}[1]{\textcolor[rgb]{0.64,0.00,0.00}{\textbf{#1}}}
\newcommand{\ExtensionTok}[1]{#1}
\newcommand{\FloatTok}[1]{\textcolor[rgb]{0.00,0.00,0.81}{#1}}
\newcommand{\FunctionTok}[1]{\textcolor[rgb]{0.13,0.29,0.53}{\textbf{#1}}}
\newcommand{\ImportTok}[1]{#1}
\newcommand{\InformationTok}[1]{\textcolor[rgb]{0.56,0.35,0.01}{\textbf{\textit{#1}}}}
\newcommand{\KeywordTok}[1]{\textcolor[rgb]{0.13,0.29,0.53}{\textbf{#1}}}
\newcommand{\NormalTok}[1]{#1}
\newcommand{\OperatorTok}[1]{\textcolor[rgb]{0.81,0.36,0.00}{\textbf{#1}}}
\newcommand{\OtherTok}[1]{\textcolor[rgb]{0.56,0.35,0.01}{#1}}
\newcommand{\PreprocessorTok}[1]{\textcolor[rgb]{0.56,0.35,0.01}{\textit{#1}}}
\newcommand{\RegionMarkerTok}[1]{#1}
\newcommand{\SpecialCharTok}[1]{\textcolor[rgb]{0.81,0.36,0.00}{\textbf{#1}}}
\newcommand{\SpecialStringTok}[1]{\textcolor[rgb]{0.31,0.60,0.02}{#1}}
\newcommand{\StringTok}[1]{\textcolor[rgb]{0.31,0.60,0.02}{#1}}
\newcommand{\VariableTok}[1]{\textcolor[rgb]{0.00,0.00,0.00}{#1}}
\newcommand{\VerbatimStringTok}[1]{\textcolor[rgb]{0.31,0.60,0.02}{#1}}
\newcommand{\WarningTok}[1]{\textcolor[rgb]{0.56,0.35,0.01}{\textbf{\textit{#1}}}}
\usepackage{graphicx}
\makeatletter
\newsavebox\pandoc@box
\newcommand*\pandocbounded[1]{% scales image to fit in text height/width
  \sbox\pandoc@box{#1}%
  \Gscale@div\@tempa{\textheight}{\dimexpr\ht\pandoc@box+\dp\pandoc@box\relax}%
  \Gscale@div\@tempb{\linewidth}{\wd\pandoc@box}%
  \ifdim\@tempb\p@<\@tempa\p@\let\@tempa\@tempb\fi% select the smaller of both
  \ifdim\@tempa\p@<\p@\scalebox{\@tempa}{\usebox\pandoc@box}%
  \else\usebox{\pandoc@box}%
  \fi%
}
% Set default figure placement to htbp
\def\fps@figure{htbp}
\makeatother
\setlength{\emergencystretch}{3em} % prevent overfull lines
\providecommand{\tightlist}{%
  \setlength{\itemsep}{0pt}\setlength{\parskip}{0pt}}
\setcounter{secnumdepth}{-\maxdimen} % remove section numbering
\usepackage{bookmark}
\IfFileExists{xurl.sty}{\usepackage{xurl}}{} % add URL line breaks if available
\urlstyle{same}
\hypersetup{
  pdftitle={Untitled},
  pdfauthor={Esteban Sierra},
  hidelinks,
  pdfcreator={LaTeX via pandoc}}

\title{Untitled}
\author{Esteban Sierra}
\date{2025-09-30}

\begin{document}
\maketitle

\begin{Shaded}
\begin{Highlighting}[]
\NormalTok{D }\OtherTok{=} \FunctionTok{read.csv}\NormalTok{(}\StringTok{\textquotesingle{}AlCorte.csv\textquotesingle{}}\NormalTok{)}
\end{Highlighting}
\end{Shaded}

\section{Análisis Descriptivo}\label{anuxe1lisis-descriptivo}

Obtener el mínimo, la mediana la media y otros valores

\begin{Shaded}
\begin{Highlighting}[]
\NormalTok{n }\OtherTok{=} \DecValTok{5} \CommentTok{\#número de variables}
\NormalTok{d }\OtherTok{=} \FunctionTok{matrix}\NormalTok{(}\ConstantTok{NA}\NormalTok{,}\AttributeTok{ncol=}\DecValTok{8}\NormalTok{,}\AttributeTok{nrow=}\NormalTok{n)}
\ControlFlowTok{for}\NormalTok{(i }\ControlFlowTok{in} \DecValTok{1}\SpecialCharTok{:}\NormalTok{n)\{}
\NormalTok{  d[i,]}\OtherTok{\textless{}{-}}\FunctionTok{c}\NormalTok{(}\FunctionTok{as.numeric}\NormalTok{(}\FunctionTok{summary}\NormalTok{(D[,i])), }\FunctionTok{sd}\NormalTok{(D[ ,i]), }\FunctionTok{sd}\NormalTok{(D[ ,i])}\SpecialCharTok{/}\FunctionTok{mean}\NormalTok{(D[ ,i]))}
\NormalTok{\}}
\NormalTok{m }\OtherTok{=} \FunctionTok{as.data.frame}\NormalTok{(d)}
\NormalTok{variables }\OtherTok{=} \FunctionTok{names}\NormalTok{(D)}
\FunctionTok{row.names}\NormalTok{(m) }\OtherTok{=}\NormalTok{ variables}
\FunctionTok{names}\NormalTok{(m) }\OtherTok{=} \FunctionTok{c}\NormalTok{(}\StringTok{"Minimo"}\NormalTok{,}\StringTok{"Q1"}\NormalTok{,}\StringTok{"Mediana"}\NormalTok{,}\StringTok{"Media"}\NormalTok{,}\StringTok{"Q3"}\NormalTok{,}\StringTok{"Máximo"}\NormalTok{,}\StringTok{"Desv Est"}\NormalTok{, }\StringTok{"CV"}\NormalTok{)}
\FunctionTok{round}\NormalTok{(m,}\DecValTok{2}\NormalTok{)}
\end{Highlighting}
\end{Shaded}

\begin{verbatim}
##             Minimo     Q1 Mediana  Media    Q3 Máximo Desv Est   CV
## Fuerza        25.0  30.00    35.0  35.00  40.0   45.0     4.55 0.13
## Potencia      45.0  60.00    75.0  75.00  90.0  105.0    13.65 0.18
## Temperatura  150.0 175.00   200.0 200.00 225.0  250.0    22.74 0.11
## Tiempo        10.0  15.00    20.0  20.00  25.0   30.0     4.55 0.23
## Resistencia   22.7  34.67    38.6  38.41  42.7   58.7     8.95 0.23
\end{verbatim}

Obtener la correlación de las variables

\begin{Shaded}
\begin{Highlighting}[]
\FunctionTok{cor}\NormalTok{(D)}
\end{Highlighting}
\end{Shaded}

\begin{verbatim}
##                Fuerza  Potencia Temperatura    Tiempo Resistencia
## Fuerza      1.0000000 0.0000000   0.0000000 0.0000000   0.1075208
## Potencia    0.0000000 1.0000000   0.0000000 0.0000000   0.7594185
## Temperatura 0.0000000 0.0000000   1.0000000 0.0000000   0.3293353
## Tiempo      0.0000000 0.0000000   0.0000000 1.0000000   0.1312262
## Resistencia 0.1075208 0.7594185   0.3293353 0.1312262   1.0000000
\end{verbatim}

Obtener el gráfico de bigote

\begin{Shaded}
\begin{Highlighting}[]
\NormalTok{colores }\OtherTok{=} \FunctionTok{rainbow}\NormalTok{(}\DecValTok{5}\NormalTok{)}
\FunctionTok{par}\NormalTok{(}\AttributeTok{mfrow=}\FunctionTok{c}\NormalTok{(}\DecValTok{1}\NormalTok{,}\DecValTok{5}\NormalTok{), }\AttributeTok{las=}\DecValTok{1}\NormalTok{)}
\FunctionTok{boxplot}\NormalTok{(D[}\DecValTok{1}\NormalTok{], }\AttributeTok{col=}\NormalTok{colores[}\DecValTok{1}\NormalTok{], }\AttributeTok{ylab=}\NormalTok{variables[}\DecValTok{1}\NormalTok{])}
\FunctionTok{boxplot}\NormalTok{(D[}\DecValTok{2}\NormalTok{], }\AttributeTok{col=}\NormalTok{colores[}\DecValTok{2}\NormalTok{], }\AttributeTok{ylab=}\NormalTok{variables[}\DecValTok{2}\NormalTok{])}
\FunctionTok{boxplot}\NormalTok{(D[}\DecValTok{3}\NormalTok{], }\AttributeTok{col=}\NormalTok{colores[}\DecValTok{3}\NormalTok{], }\AttributeTok{ylab=}\NormalTok{variables[}\DecValTok{3}\NormalTok{])}
\FunctionTok{boxplot}\NormalTok{(D[}\DecValTok{4}\NormalTok{], }\AttributeTok{col=}\NormalTok{colores[}\DecValTok{4}\NormalTok{], }\AttributeTok{ylab=}\NormalTok{variables[}\DecValTok{4}\NormalTok{])}
\FunctionTok{boxplot}\NormalTok{(D[}\DecValTok{5}\NormalTok{], }\AttributeTok{col=}\NormalTok{colores[}\DecValTok{5}\NormalTok{], }\AttributeTok{ylab=}\NormalTok{variables[}\DecValTok{5}\NormalTok{])}
\end{Highlighting}
\end{Shaded}

\pandocbounded{\includegraphics[keepaspectratio]{main_files/figure-latex/unnamed-chunk-4-1.pdf}}

\section{Obtener mejor modelo de
regresión}\label{obtener-mejor-modelo-de-regresiuxf3n}

\subsection{Criterio AIC}\label{criterio-aic}

\begin{Shaded}
\begin{Highlighting}[]
\NormalTok{R }\OtherTok{=} \FunctionTok{lm}\NormalTok{(Resistencia }\SpecialCharTok{\textasciitilde{}}\NormalTok{ . , }\AttributeTok{data =}\NormalTok{ D)}
\FunctionTok{step}\NormalTok{(R, }\AttributeTok{direction=}\StringTok{"both"}\NormalTok{, }\AttributeTok{trace=}\DecValTok{1}\NormalTok{)}
\end{Highlighting}
\end{Shaded}

\begin{verbatim}
## Start:  AIC=102.96
## Resistencia ~ Fuerza + Potencia + Temperatura + Tiempo
## 
##               Df Sum of Sq     RSS    AIC
## - Fuerza       1     26.88  692.00 102.15
## - Tiempo       1     40.04  705.16 102.72
## <none>                      665.12 102.96
## - Temperatura  1    252.20  917.32 110.61
## - Potencia     1   1341.01 2006.13 134.08
## 
## Step:  AIC=102.15
## Resistencia ~ Potencia + Temperatura + Tiempo
## 
##               Df Sum of Sq     RSS    AIC
## - Tiempo       1     40.04  732.04 101.84
## <none>                      692.00 102.15
## + Fuerza       1     26.88  665.12 102.96
## - Temperatura  1    252.20  944.20 109.47
## - Potencia     1   1341.02 2033.02 132.48
## 
## Step:  AIC=101.84
## Resistencia ~ Potencia + Temperatura
## 
##               Df Sum of Sq     RSS    AIC
## <none>                      732.04 101.84
## + Tiempo       1     40.04  692.00 102.15
## + Fuerza       1     26.88  705.16 102.72
## - Temperatura  1    252.20  984.24 108.72
## - Potencia     1   1341.01 2073.06 131.07
\end{verbatim}

\begin{verbatim}
## 
## Call:
## lm(formula = Resistencia ~ Potencia + Temperatura, data = D)
## 
## Coefficients:
## (Intercept)     Potencia  Temperatura  
##    -24.9017       0.4983       0.1297
\end{verbatim}

\subsection{Criterio BIC}\label{criterio-bic}

\begin{Shaded}
\begin{Highlighting}[]
\NormalTok{n }\OtherTok{=} \FunctionTok{length}\NormalTok{(D)}
\NormalTok{R }\OtherTok{=} \FunctionTok{lm}\NormalTok{(Resistencia }\SpecialCharTok{\textasciitilde{}}\NormalTok{ . , }\AttributeTok{data =}\NormalTok{ D)}
\FunctionTok{step}\NormalTok{(R, }\AttributeTok{direction=}\StringTok{"both"}\NormalTok{, }\AttributeTok{k=}\FunctionTok{log}\NormalTok{(n))}
\end{Highlighting}
\end{Shaded}

\begin{verbatim}
## Start:  AIC=101.01
## Resistencia ~ Fuerza + Potencia + Temperatura + Tiempo
## 
##               Df Sum of Sq     RSS    AIC
## - Fuerza       1     26.88  692.00 100.59
## <none>                      665.12 101.01
## - Tiempo       1     40.04  705.16 101.16
## - Temperatura  1    252.20  917.32 109.05
## - Potencia     1   1341.01 2006.13 132.52
## 
## Step:  AIC=100.59
## Resistencia ~ Potencia + Temperatura + Tiempo
## 
##               Df Sum of Sq     RSS    AIC
## <none>                      692.00 100.59
## - Tiempo       1     40.04  732.04 100.67
## + Fuerza       1     26.88  665.12 101.01
## - Temperatura  1    252.20  944.20 108.30
## - Potencia     1   1341.02 2033.02 131.31
\end{verbatim}

\begin{verbatim}
## 
## Call:
## lm(formula = Resistencia ~ Potencia + Temperatura + Tiempo, data = D)
## 
## Coefficients:
## (Intercept)     Potencia  Temperatura       Tiempo  
##    -30.0683       0.4983       0.1297       0.2583
\end{verbatim}

\begin{Shaded}
\begin{Highlighting}[]
\FunctionTok{extractAIC}\NormalTok{(R, }\AttributeTok{k=}\FunctionTok{log}\NormalTok{(n))}
\end{Highlighting}
\end{Shaded}

\begin{verbatim}
## [1]   5.0000 101.0102
\end{verbatim}

\subsection{Criterio HQC}\label{criterio-hqc}

\begin{Shaded}
\begin{Highlighting}[]
\NormalTok{HQC }\OtherTok{=} \FunctionTok{step}\NormalTok{(R, }\AttributeTok{direction=}\StringTok{"both"}\NormalTok{, }\AttributeTok{k=}\DecValTok{2}\SpecialCharTok{*}\FunctionTok{log}\NormalTok{(}\FunctionTok{log}\NormalTok{(n)))}
\end{Highlighting}
\end{Shaded}

\begin{verbatim}
## Start:  AIC=97.72
## Resistencia ~ Fuerza + Potencia + Temperatura + Tiempo
## 
##               Df Sum of Sq     RSS     AIC
## <none>                      665.12  97.722
## - Fuerza       1     26.88  692.00  97.959
## - Tiempo       1     40.04  705.16  98.524
## - Temperatura  1    252.20  917.32 106.415
## - Potencia     1   1341.01 2006.13 129.890
\end{verbatim}

\subsection{Significancia}\label{significancia}

\begin{Shaded}
\begin{Highlighting}[]
\NormalTok{BestModel }\OtherTok{=} \FunctionTok{lm}\NormalTok{(Resistencia }\SpecialCharTok{\textasciitilde{}}\NormalTok{ Fuerza }\SpecialCharTok{+}\NormalTok{ Potencia }\SpecialCharTok{+}\NormalTok{ Temperatura, }\AttributeTok{data =}\NormalTok{ D)}
\FunctionTok{summary}\NormalTok{(BestModel)}
\end{Highlighting}
\end{Shaded}

\begin{verbatim}
## 
## Call:
## lm(formula = Resistencia ~ Fuerza + Potencia + Temperatura, data = D)
## 
## Residuals:
##      Min       1Q   Median       3Q      Max 
## -12.3817  -2.6421  -0.5942   3.1892   8.4017 
## 
## Coefficients:
##              Estimate Std. Error t value Pr(>|t|)    
## (Intercept) -32.31000   12.52410  -2.580  0.01589 *  
## Fuerza        0.21167    0.21261   0.996  0.32864    
## Potencia      0.49833    0.07087   7.032 1.82e-07 ***
## Temperatura   0.12967    0.04252   3.049  0.00522 ** 
## ---
## Signif. codes:  0 '***' 0.001 '**' 0.01 '*' 0.05 '.' 0.1 ' ' 1
## 
## Residual standard error: 5.208 on 26 degrees of freedom
## Multiple R-squared:  0.6967, Adjusted R-squared:  0.6617 
## F-statistic: 19.91 on 3 and 26 DF,  p-value: 6.507e-07
\end{verbatim}

\begin{Shaded}
\begin{Highlighting}[]
\CommentTok{\# Economía de las variables}
\CommentTok{\#Significación global (Prueba para el modelo)}
\CommentTok{\#Significación individual (Prueba para cada 𝛽𝑖)}
\CommentTok{\#Variación explicada por el modelo}
\end{Highlighting}
\end{Shaded}

\subsubsection{Economía de las
variables}\label{economuxeda-de-las-variables}

\subsubsection{Significancia global}\label{significancia-global}

La significancia global en este modelo es alta ya que el p-value es
menor a 0.05.

\subsubsection{Significancia individual}\label{significancia-individual}

El modelo tiene una alta significancia porque - Potencia: t = 7.033 y
valor p casi cero -- Temperatura: t = 3.050 y valor p = 0.00499

\subsubsection{Variación explicada por el
modelo}\label{variaciuxf3n-explicada-por-el-modelo}

\begin{Shaded}
\begin{Highlighting}[]
\FunctionTok{confint}\NormalTok{(BestModel)}
\end{Highlighting}
\end{Shaded}

\begin{verbatim}
##                    2.5 %     97.5 %
## (Intercept) -58.05364728 -6.5663527
## Fuerza       -0.22535738  0.6486907
## Potencia      0.35265865  0.6440080
## Temperatura   0.04226186  0.2170715
\end{verbatim}

\section{Análisis de validez del modelo
encontrado}\label{anuxe1lisis-de-validez-del-modelo-encontrado}

\subsection{Análisis de residuos}\label{anuxe1lisis-de-residuos}

\subsubsection{Homocedasticidad}\label{homocedasticidad}

\subsubsection{Independencia}\label{independencia}

\section{A1 Regresión múltiple}\label{a1-regresiuxf3n-muxfaltiple}

\begin{enumerate}
\def\labelenumi{\arabic{enumi}.}
\tightlist
\item
  Haz un análisis descriptivo de los datos: medidas principales y
  gráficos
\item
  Encuentra el mejor modelo de regresión que explique la variable
  Resistencia. Analiza el modelo basándote en:
\item
  Significancia del modelo: 1. Economía de las variables 2.
  Significación global (Prueba para el modelo) 3. Significación
  individual (Prueba para cada 𝛽𝑖) 4. Variación explicada por el modelo
\item
  Analiza la validez del modelo encontrado:
\item
  Análisis de residuos (homocedasticidad, independencia, etc)
\item
  No multicolinealidad de Xi
\item
  Emite conclusiones sobre el modelo final encontrado e interpreta en el
  contexto del problema el efecto de las variables predictoras en la
  variable respuesta
\end{enumerate}

\end{document}
